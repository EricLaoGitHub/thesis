\chapter{Challenges}

Although most of the recent works focus on simulated annealing algorithm cite{00}, we believe that giving more control to the designer and using his interventions to set some constraints yields good analog placement results. Our approach introduces a semi-automatic analog placement approach guided by the designer's preferences. This semi-automatic approach also helps the designer to debug more efficiently and make adjustments easier since he will, himself, control the overall relative placement of the circuit but at the same time, some tiresome and error-prone tasks are automated.
\newline 
\newline 
\indent Another critical part in analog design, that has not been considered in most of the previous work, is the routing phase. Conventionally, the execution of placement and routing has been sequential. If the routing is a two-step procedure, the execution of the global routing and detailed routing is also sequential. Simultaneous performance optimization of the placement and the global routing can lead to a more accurate search and this is what we aim to do unlike most of the recent research who only consider the placement alone. The general reluctance of analog designers toward using automatic layout generation tools is because they are skeptical about their reliability and they would like to have full control over the layout generation process. They would prefer the layout tool to be an assistant to them and generating a template for the layout is sometimes better for them.
