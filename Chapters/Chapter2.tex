\chapter{Problem Definition}
An automatic placement tool should produce analog device-level layouts similar in density and performance to the high-quality manual layouts. Having the devices for the selected topology sized, they must be laid out in the chip. Considering these devices and their interconnects, the problem is to explore non-overlap placements of given modules. The capability to deal with layout constraints, in order to eliminate unwanted parasitics due to the process variations, is mandatory. cite{00} enumerates the most common placement constraints that are respected by modern analog placer. Among them, symmetries and symmetry-island cite{3} are the most used constraints. Other constraints like common-centroid cite{4}, proximity and range are also often considered cite{5}. Taking into account current and signal path improves performance accuracy cite{6}. Devices can be placed depending on their thermal impact on the chip cite{7}. Regularity cite{8} and boundary constraints cite{9} enhance routability and suppress parasitics induced by extra bends of wires and via cost.
\newline 
\newline 
\indent Given a placement of a set of layout devices together with their positions, the layout must then be interconnected according to the netlist in a design rule correct way. Due to the fact that the performances of an analog circuit are critically dependent on layout parasitics, the routing of an analog circuit requires more attention than that of a digital circuit. Analog routing constraints can be critical such as symmetry, topology and wirelength matching.
\newline 
\newline 
\indent Recent research focused on using simulated annealing algorithm (SA) in combination with topological representations to respect these placement constraints. Topological representations encode the positioning relations between devices and SA optimizer alters their relative positions. The most popular representations used in analog placement over the last decades are Sequence-Pair cite{10}, B*-Tree cite{11}, Transitive Closure Graph cite{12}, Ordered-Tree cite{13} and slicing floorplans cite{15} and they were coupled with some constraints to respect at the same time. This will be discussed in the following section
% \newline 
% \newline 
% \indent Although most of the recent works focus on simulated annealing algorithm cite{00}, we believe that giving more control to the designer and using his interventions to set some constraints yields good analog placement results. Our approach introduces a semi-automatic analog placement approach guided by the designer's preferences. This semi-automatic approach also helps the designer to debug more efficiently and make adjustments easier since he will, himself, control the overall relative placement of the circuit but at the same time, some tiresome and error-prone tasks are automated.